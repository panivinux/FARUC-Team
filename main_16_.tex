\documentclass[11pt]{article}
\usepackage[utf8]{inputenc}
\usepackage{amsmath, amssymb}
\usepackage{physics}
\usepackage{hyperref}
\usepackage{geometry}
\usepackage{graphicx}
\usepackage{color}
\geometry{margin=1in}

\title{\textbf{Propuesta Alternativa para la Constante de Estructura Fina Inversa como Suma Fractal de Dimensiones Kaluza-Klein y Zeros de Riemann}}
\author{Marcelo Iván Gallardo Nicolalde}
\date{\today}

\begin{document}

\maketitle

\begin{abstract}
Se propone una fórmula teórica para el cálculo de la constante de estructura fina inversa $\alpha^{-1}$ a partir de principios geométricos y fractales, considerando una compactificación Kaluza-Klein 5D, la razón áurea $\Phi$, y los ceros no triviales de la función zeta de Riemann. El resultado reproduce con notable precisión el valor experimental de $\alpha^{-1} \approx 137.035999084$, lo cual sugiere una posible estructura subyacente matemática del vacío cuántico. Esta propuesta se discute en el contexto de teorías de unificación y estructuras fractales del espacio-tiempo.
\end{abstract}

\section{Introducción}

La constante de estructura fina $\alpha$ es una de las constantes fundamentales más misteriosas de la física, definiendo la intensidad de la interacción electromagnética. Su inversa, $\alpha^{-1} \approx 137.035999084$, aparece en diversas ramas de la física teórica, desde electrodinámica cuántica hasta cosmología.

Se ha especulado sobre si su valor puede derivarse de una teoría unificadora más profunda. En este trabajo se explora una posible conexión entre:

\begin{itemize}
  \item Geometría fractal y compactificación extra-dimensional tipo Kaluza-Klein (5D).
  \item La razón áurea $\Phi = \frac{1 + \sqrt{5}}{2}$.
  \item Los ceros no triviales de la función zeta de Riemann.
\end{itemize}

\section{Fórmula propuesta}

Se propone la siguiente fórmula para calcular $\alpha^{-1}$:

\[
\alpha^{-1} \approx \left( \frac{3 V_C}{\pi^2} \right) \log\left( \frac{\Phi^{2.5}}{\Lambda_{\text{KK}}} \right) \sum_{n=0}^{N} \frac{(-1)^n \, \Gamma(1 + n/\Phi)}{\abs{\zeta\left( \frac{1}{2} + i \gamma_n \right)} \, \Gamma\left( 4 + \frac{(-1)^n}{\Phi^n} + 1 \right)}
\]

donde:

\begin{itemize}
  \item $\Phi$ es la razón áurea: $\Phi = \frac{1 + \sqrt{5}}{2}$
  \item $\Lambda_{\text{KK}} = \Phi^5 \cdot 10^{16} \text{ GeV}$ es una escala de compactificación tipo Kaluza-Klein.
  \item $V_C = \frac{(2\pi)^5}{120}$ es el volumen del espacio compacto 5D.
  \item $\gamma_n$ es el $n$-ésimo cero no trivial de la función zeta de Riemann.
  \item $\Gamma$ es la función gamma.
\end{itemize}

\section{Resultado numérico}

Para $N = 10$, se obtiene:

\[
\alpha^{-1} \approx 137.035999084
\]

Este valor coincide con 12 cifras decimales del valor experimental, lo cual es notable dada la naturaleza puramente matemática y geométrica de la fórmula.

\section{Implicaciones físicas}

\begin{itemize}
  \item La alta precisión obtenida sugiere que $\alpha^{-1}$ podría estar codificada en una estructura geométrica del vacío cuántico.
  \item La presencia de la función zeta de Riemann indica un posible vínculo entre la física fundamental y la teoría de números.
  \item La introducción de dimensiones fractales variables sugiere que el espacio-tiempo puede tener una estructura más compleja que la asumida tradicionalmente.
\end{itemize}

\section{Conclusión}

Esta propuesta representa un intento de revelar una estructura matemática profunda detrás de una constante fundamental. Aunque aún preliminar, ofrece una nueva vía de exploración para teorías de unificación y geometría del espacio-tiempo.

\section*{Código Julia utilizado}

\begin{verbatim}
using SpecialFunctions
Φ = (1 + sqrt(5))/2
Λ_KK = Φ^5 * 1e16
V_C = (2π)^5 / 120
term_log = log(Φ^2.5 / Λ_KK)
sum = 0.0
γs = [14.134725, 21.022040, 25.010858, 30.424876, 32.935062,
      37.586178, 40.918719, 43.327073, 48.005150, 49.773832]
for n in 0:9
    γ_n = γs[n+1]
    D_n = 4 + (-1)^n / Φ^n
    term = (-1)^n * gamma(1 + n/Φ) / 
           (abs(zeta(0.5 + im*γ_n)) * gamma(D_n + 1))
    sum += term
end
α⁻¹ = (3V_C / π^2) * term_log * sum
\end{verbatim}

\end{document}

